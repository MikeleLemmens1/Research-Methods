%==============================================================================
% Paper Research Methods: onderzoeksvoorstel
%==============================================================================

\documentclass{hogent-article}

\usepackage{lipsum} % Voor vultekst
\usepackage[backend=biber]{biblatex} 

% Invoegen bibliografiebestand
\addbibresource{LemmensMikele2024RM.bib}

% Informatie over de opleiding, het vak en soort opdracht
\studyprogramme{Professionele bachelor toegepaste informatica}
\course{Research Methods}
\assignmenttype{Paper: Onderzoeksvoorstel}
\academicyear{2023-2024}

\title{Optimalisatie van het voorstellen van recepten op basis van gebruikersvoorkeuren}

\author{Mikele Lemmens}
\email{Mikele.lemmens@student.hogent.be}

\projectrepo{https://github.com/hogenttin/rm-2324-lemmensmikele}

\specialisation{AI \& Data Engineering}

\keywords{recept voorstellen, voorkeur voeding, ingrediënt matching}

\begin{document}

\begin{abstract}
    
Een dagelijkse warme maaltijd is een belangrijk deel van een evenwichtig eetpatroon. In deze studie wordt een toepassing ontwikkeld die een relevant recept voorstelt aan een gebruiker die hulp wil bij het bedenken ervan en belang hecht aan een gezonde eetgewoonte. Er wordt geen gebruik gemaakt van taalmodellen om een bereidingswijze te genereren, maar wel van een dataset van bestaande hoofdgerechten. Het selecteren hiervan maakt ook deel uit van het onderzoek.

De relevantie van een gerecht wordt bepaald op basis van een kwantitatieve waarderingsscore die de gebruiker aan voorgestelde gerechten geeft. Er worden twee technieken gecombineerd; In de eerste wordt een lijst van ingrediënten opgesteld die voorkomen in een dataset van recepten uit verschillende categorieën. Voor die verzameling wordt een score berekend die een maat is voor hoe goed de elementen bij elkaar passen. Hoe hoger de score, hoe groter de kans dat een gevonden recept bestaande uit deze ingrediënten een goede waardering krijgt. De gebruiker geeft een gewenst ingrediënt in waarna een viertal andere worden aangevuld door de toepassing.

De tweede techniek berekent een waarde die aangeeft in hoeverre een reeks gerechten die op opeenvolgende dagen (zouden) worden klaargemaakt aan elkaar verwant zijn. Hoe lager deze score, hoe meer variëteit tussen de gerechten. Er wordt verwacht dat een grotere variatie gecorreleerd is met de waardering die elk recept in de reeks krijgt wanneer deze worden aangereikt. De recepten zijn geselecteerd op basis van de eerder gegenereerde ingrediëntenlijst en een gekozen categorie.

Het verwachte resultaat is dat de waardering van voorgestelde maaltijden m.b.v. de beschreven werkwijze hoger is dan diegene die voortkomt uit een receptenlijst opgesteld door een eenvoudige zoekopdracht met sleutelwoorden. Het gebruik van deze toepassing dient zo te zorgen voor een aangenamere kookervaring, aangepast aan de gebruiker en zonder de noodzaak om zelf een hoofdgerecht te bedenken.


\end{abstract}

\tableofcontents

\bigskip

% TODO: Neem je dit jaar ook de bachelorproef op? Haal dan de tekst hieronder
% uit commentaar en pas aan voor jouw situatie.

%\paragraph{Opmerking}

% Ik neem dit jaar ook de bachelorproef op. De inhoud van dit onderzoeksvoorstel dient ook als het onderwerpvoor mijn bachelorproef. Mijn promotor is (Mr./Mevr.) X.\ Familienaam.

% TODO: Beschrijf de eventuele verschillen en/of verbeteringen in dit document t.o.v.\ jouw onderzoeksvoorstel dat je ingediend hebt voor de bachelorproef.

\section{Inleiding}%
\label{sec:inleiding}

% TODO: (fase 1 - onderzoeksvraag formuleren)

Waarover zal het onderzoek gaan? Introduceer het thema en zorg dat volgende zaken zeker duidelijk aanwezig zijn:

\begin{itemize}
  \item kaderen thema
  \item de doelgroep
  \item de probleemstelling en onderzoeksvraag
  \item de onderzoeksdoelstelling
\end{itemize}

Denk er aan: een typische bachelorproef is \textit{toegepast onderzoek}, wat betekent dat je start vanuit een concrete probleemsituatie in bedrijfscontext, een \textbf{casus}. Het is belangrijk om je onderwerp goed af te bakenen: je gaat voor die \textit{ene specifieke probleemsituatie} op zoek naar een goede oplossing, op basis van de huidige kennis in het vakgebied.

De doelgroep moet ook concreet en duidelijk zijn, dus geen algemene of vaag gedefinieerde groepen zoals \emph{bedrijven}, \emph{developers}, \emph{Vlamingen}, enz. Je richt je in elk geval op it-professionals, een bachelorproef is geen populariserende tekst. Eén specifiek bedrijf (die te maken hebben met een concrete probleemsituatie) is dus beter dan \emph{bedrijven} in het algemeen.

Formuleer duidelijk de onderzoeksvraag! De begeleiders lezen nog steeds te veel voorstellen waarin we geen onderzoeksvraag terugvinden.

Waarom is het nuttig om dit onderwerp te onderzoeken? Wat is de onderzoeksdoelstelling (formuleer deze S.M.A.R.T.)? Wat wil je precies bereiken? Wat zie je als het concrete eindresultaat van je onderzoek, naast de uitgeschreven scriptie? Is het een proof-of-concept, een prototype, een rapport met aanbevelingen, \ldots Met welk eindresultaat kan je je bachelorproef als een succes beschouwen?

\section{Literatuurstudie}%
\label{sec:literatuurstudie}

% TODO: (fase 3, 4 - literatuurstudie)

Hier beschrijf je de \emph{state-of-the-art} rondom je gekozen onderzoeksdomein, d.w.z.\ een inleidende, doorlopende tekst over het onderzoeksdomein van je bachelorproef. Je steunt daarbij heel sterk op de professionele \emph{vakliteratuur}, en niet zozeer op populariserende teksten voor een breed publiek. Wat is de huidige stand van zaken in dit domein, en wat zijn nog eventuele open vragen (die misschien de aanleiding waren tot je onderzoeksvraag!)? 

Je mag deze sectie nog verder onderverdelen in subsecties als dit de structuur van de tekst kan verduidelijken.

Zijn er al gelijkaardige onderzoeken gevoerd? Wat concluderen ze? Wat is het verschil met jouw onderzoek?

Verwijs bij elke introductie van een term of bewering over het domein naar de vakliteratuur! Denk zeker goed na welke werken je refereert en waarom.

Draag zorg voor correcte literatuurverwijzingen! Een bronvermelding hoort thuis \emph{binnen} de zin waar je je op die bron baseert, dus niet er buiten, bijvoorbeeld~\autocite{Hykes2013}! Maak meteen een verwijzing als je gebruik maakt van een bron. Doe dit dus \emph{niet} aan het einde van een lange paragraaf. Baseer nooit teveel aansluitende tekst op eenzelfde bron.

Als je informatie over bronnen verzamelt in JabRef, zorg er dan voor dat alle nodige info aanwezig is om de bron terug te vinden (zoals uitvoerig besproken in de lessen Research Methods).

% Refereren naar de literatuur kan met:
% \autocite{BIBTEXKEY} => (Auteur, jaartal): voor een referentie tussen
% haakjes, waar de auteursnaam GEEN onderdeel is van een zin.
% \textcite{BIBTEXKEY} => Auteur (jaartal): voor een narratieve referentie,
% waar de naam van de auteur effectief een onderdeel is van de zin.

\section{Methodologie}%
\label{sec:methodologie}

% TODO: (fase 5 - methodologie)

Hier beschrijf je hoe je van plan bent het onderzoek te voeren. Verdeel het onderzoek op in verschillende fasen en probeer te formuleren welke concrete deliverable(s) het resultaat zijn van elke fase.

Welke onderzoekstechniekenen ga je toepassen om elk van je onderzoeksvragen te beantwoorden? Gebruik je hiervoor literatuurstudie, interviews met belanghebbenden (bv.\ voor re\-quire\-ments-a\-na\-ly\-se), experimenten, simulaties, vergelijkende studie, risico-analyse, PoC, \ldots?

Valt je onderwerp onder één van de typische soorten bachelorproeven die besproken zijn in de lessen Research Methods (bv.\ vergelijkende studie of risico-analyse)? Zorg er dan ook voor dat we duidelijk de verschillende stappen terug vinden die we verwachten in dit soort onderzoek!

Vermijd onderzoekstechnieken die geen objectieve, meetbare resultaten kunnen opleveren. Enquêtes, bijvoorbeeld, zijn voor een bachelorproef informatica meestal \textbf{niet geschikt}. De antwoorden zijn eerder meningen dan feiten en in de praktijk blijkt het ook bijzonder moeilijk om voldoende respondenten te vinden. Studenten die een enquête willen voeren, hebben meestal ook geen goede definitie van de populatie, waardoor ook niet kan aangetoond worden dat eventuele resultaten representatief zijn.

Uit dit onderdeel moet duidelijk naar voor komen dat je bachelorproef ook technisch voldoen\-de diepgang zal bevatten. Het zou niet kloppen als een bachelorproef informatica ook door bv.\ een student marketing zou kunnen uitgevoerd worden.

Je beschrijft ook al welke tools (hardware, software, diensten, \ldots) je denkt hiervoor te gebruiken of te ontwikkelen.

Probeer ook een tijdschatting te maken door een deadline op te geven voor elke fase. Neem voldoende tijd voor de belangrijkste fasen in je onderzoek, nl.\ het uitwerken van je eigen bijdrage (PoC bouwen, experimenten uitvoeren, enz.). Hou er rekening mee dat je typisch één dag per week kan werken aan je bachelorproef. Dat betekent dat uitspraken als ``voor deze fase wordt twee weken tijd voorzien'' erg dubbelzinnig zijn. Betekent dit dat je in realiteit twee dagen zal werken aan deze fase? Of tien werkdagen verspreid over een aantal weken? Zorg dat het duidelijk is wat je precies bedoelt!

\section{Verwachte resultaten}%
\label{sec:verwachte-resultaten}

% TODO: (fase 6 - afwerking)

Hier beschrijf je welke resultaten je verwacht en waarom. Bijvoorbeeld, volgens je literatuuronderzoek is softwarepakket A het meest gebruikte en dus denk je dat het voor deze casus ook het meest geschikt zal zijn. Natuurlijk kan je niet in de toekomst kijken en mag je geen alternatieve mogelijkheden uitsluiten.

Als je experimenten, simulaties of metingen uitvoert, kan je overwegen om een mock-up te maken van een grafiek van de uitkomst die je vermoed. Benoem zeker al je assen en meeteenheden die je gaat gebruiken. Hierdoor krijg je ook een concreet beeld van het soort data je zal moeten verzamelen. Pas hierbij toe wat je in Data Science \& AI geleerd hebt over visualisatie van data en toepassen van correcte statistische technieken.

\section{Discussie, verwachte conclusie}%
\label{sec:discussie-conclusie}

Wat heeft de doelgroep van je onderzoek aan het resultaat? Op welke manier biedt jouw onderzoek een meerwaarde?

Het is \textbf{niet} erg indien uit je onderzoek andere resultaten en conclusies vloeien dan dat je hier beschrijft: het is dan juist interessant om te onderzoeken waarom jouw hypothesen niet overeenkomen met de resultaten.

Als je onderwerp zich daartoe leent, kan je eventueel ook suggesties doen voor een vervolg, hetzij verder onderzoek, hetzij verder bouwen op een PoC of prototype tot een eindproduct, hetzij mogelijkheden om de resultaten te valoriseren of commercialiseren.

%------------------------------------------------------------------------------
% Referentielijst
%------------------------------------------------------------------------------
% TODO: (fase 4) de gerefereerde werken moeten in BibTeX-bestand
% bibliografie.bib voorkomen. Gebruik JabRef om je bibliografie bij te
% houden.

%\printbibliography[heading=bibintoc, title={References}] 

\end{document}